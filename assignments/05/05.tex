%	
%	13-2 - Week 5, deadline on 6th of July 2013
%	

\documentclass[11pt,english]{article}

\usepackage[utf8]{inputenc}
\usepackage{fancyhdr}
\usepackage{sectsty}
\usepackage{amsmath,amssymb}	% for mathematical notation
\usepackage[linesnumbered]{algorithm2e}

%========== meta data ==========%

\title
{
	\vspace{1in}
	Algorithms \& Datastructures\\
	\huge Assignment 5
}

\author
{
	Casper B. Hansen\\
	\small Department of Computer Science\\
	\small The University of Copenhagen\\
	\texttt{fvx507@alumni.ku.dk}
	\and
	Hans J. T. Stephensen\\
	\small Department of Computer Science\\
	\small The University of Copenhagen\\
	\texttt{xkv467@alumni.ku.dk}
}

\date{\today}


%========== settings ==========%

\setlength{\headheight}{15pt}
\sectionfont{\Large}


%========== macros ==========%

% no macros yet


%========== document ==========%

\begin{document}

\clearpage
\maketitle
\thispagestyle{empty}

%========== mandatory ==========%

\newpage
\pagestyle{fancy}

\section*{Mandatory Hand-ins}

\subsection*{13-3 AVL trees}
\large{An AVL tree is a binary tree that is \textit{height balanced}; for each
node $x$, the heights of the left and right subtrees of $x$ differs by at most
1. To implement an AVL tree, we maintain an extra attribute in each node $x.h$,
the height of node $x$. As for any other binary search tree $T$, we assume
that $T.root$ points to the root node.}
\\\\
\noindent \large{\textbf{a} \mdseries Prove that an AVL tree with $n$ nodes
has height $O(lg n)$. (Hint: Prove that an AVL tree of height $h$ has at
least $F_h$ nodes, where $F_h$ is the $h$th fibonacci number.)}
\\
...
\\\\
\noindent \large{\textbf{b} \mdseries To insert into an AVL tree, we first
place a node into the appropriate place in binary search tree order.
Afterward, the tree might no longer be height balanced. Specifically, the
heights of the left and right children of some node might differ by 2.
Describe a procedure \texttt{Balance}, which takes a subtree rooted at $x$
whose left and right children are height balanced and have heights that differ
by at most 2, i.e., $|x.right.h - x.left.h| \leq 2$, and that alters the
subtree rooted at $x$ to be height balanced. (Hint: Use rotations)}
\\
...
\\\\
\noindent \large{\textbf{c} \mdseries Using part (b), describe a recursive
procedure \texttt{AVL-Insert($x$,$z$)} that takes a node $x$ within an AVL
tree and a newly created node $z$ (whose key has already been filled in), and
adds $z$ to the subtree rooted at $x$, maintaining the property that $x$ is
the root of an AVL tree. As in \texttt{Tree-Insert} from section 12.3, assume
that $z.key$ has already been filled in and that $z.left =$ \texttt{NIL} and
$z.right =$ \texttt{NIL}; also assume that $z.h = 0$. Thus, to insert the node
$z$ into an AVL tree $T$, we call \texttt{AVL-Insert($T.root$,$z$)}.}
\\
...
\\\\
\noindent \large{\textbf{d} \mdseries Show that \texttt{AVL-Insert}, run on an
$n$-node AVL tree, takes $O(lg n)$ time and performs $O(1)$ rotations.}
\\
...


%========== optional ==========%

% no optional hand-ins


%========== extras ==========%

% no extra hand.ins

%========== exercises ==========%

% \newpage
% \pagestyle{fancy}
% \section*{Exercises}
% ...

\end{document}

