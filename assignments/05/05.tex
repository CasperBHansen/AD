%	
%	13-2 - Week 5, deadline on 6th of July 2013
%	

\documentclass[11pt,english]{article}

\usepackage[utf8]{inputenc}
\usepackage{fancyhdr}
\usepackage{sectsty}
\usepackage{amsmath,amssymb}	% for mathematical notation
\usepackage[linesnumbered]{algorithm2e}

%========== meta data ==========%

\title
{
	\vspace{1in}
	Algorithms \& Datastructures\\
	\huge Assignment 5
}

\author
{
	Casper B. Hansen\\
	\small Department of Computer Science\\
	\small The University of Copenhagen\\
	\texttt{fvx507@alumni.ku.dk}
	\and
	Hans J. T. Stephensen\\
	\small Department of Computer Science\\
	\small The University of Copenhagen\\
	\texttt{xkv467@alumni.ku.dk}
}

\date{\today}


%========== settings ==========%

\setlength{\headheight}{15pt}
\sectionfont{\Large}


%========== macros ==========%

% no macros yet


%========== document ==========%

\begin{document}

\clearpage
\maketitle
\thispagestyle{empty}

%========== mandatory ==========%

\newpage
\pagestyle{fancy}

\section*{Mandatory Hand-ins}

\subsection*{13-3 AVL trees}
\large{An AVL tree is a binary tree that is \textit{height balanced}; for each
node $x$, the heights of the left and right subtrees of $x$ differs by at most
1. To implement an AVL tree, we maintain an extra attribute in each node $x.h$,
the height of node $x$. As for any other binary search tree $T$, we assume
that $T.root$ points to the root node.}
\\\\
\noindent \large{\textbf{a} \mdseries Prove that an AVL tree with $n$ nodes
has height $O(lg n)$. (Hint: Prove that an AVL tree of height $h$ has at
least $F_h$ nodes, where $F_h$ is the $h$th fibonacci number.)}
...

%========== optional ==========%

% no optional hand-ins


%========== extras ==========%

% no extra hand.ins

%========== exercises ==========%

% \newpage
% \pagestyle{fancy}
% \section*{Exercises}
% ...

\end{document}

