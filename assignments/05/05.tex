%	
%	13-2 - Week 5, deadline on 6th of July 2013
%	

\documentclass[11pt,english]{article}

\usepackage[utf8]{inputenc}
\usepackage{fancyhdr}
\usepackage{sectsty}
\usepackage{amsmath,amssymb}	% for mathematical notation
\usepackage[linesnumbered]{algorithm2e}

%========== meta data ==========%

\title
{
	\vspace{1in}
	Algorithms \& Datastructures\\
	\huge Assignment 5
}

\author
{
	Casper B. Hansen\\
	\small Department of Computer Science\\
	\small The University of Copenhagen\\
	\texttt{fvx507@alumni.ku.dk}
	\and
	Hans J. T. Stephensen\\
	\small Department of Computer Science\\
	\small The University of Copenhagen\\
	\texttt{xkv467@alumni.ku.dk}
}

\date{\today}


%========== settings ==========%

\setlength{\headheight}{15pt}
\sectionfont{\Large}


%========== macros ==========%

% no macros yet


%========== document ==========%

\begin{document}

\clearpage
\maketitle
\thispagestyle{empty}

%========== mandatory ==========%

\newpage
\pagestyle{fancy}

\section*{Mandatory Hand-ins}

\subsection*{13-3 AVL trees}
\large{An AVL tree is a binary tree that is \textit{height balanced}; for each
node $x$, the heights of the left and right subtrees of $x$ differs by at most
1. To implement an AVL tree, we maintain an extra attribute in each node $x.h$,
the height of node $x$. As for any other binary search tree $T$, we assume
that $T.root$ points to the root node.}
\\\\
\noindent \large{\textbf{a} \mdseries Prove that an AVL tree with $n$ nodes
has height $O(lg n)$. (Hint: Prove that an AVL tree of height $h$ has at
least $F_h$ nodes, where $F_h$ is the $h$th fibonacci number.)}
\\
By induction, we will hereby prove that the number of elements in an AVL tree of 
depth $h$, is lower bounded by the h'th Fibonacci number $F_h$. Vi first note 
that in order to be sure we lower bound every possible tree, we must look at 
the smallest possible AVL tree of arbitrary height. In order to achieve this,
we need to assume that the difference in height between subtrees, are as big
as possible.\\[5pt]

\noindent \textbf{Base cases} \\[5pt]
By definition, the zero'th and first Fibonacci is 0 and 1 respectivly, 
corresponding to the empty AVL tree with height 0 and an AVL tree with 1 
element and thus height 1. Using the Fibonacci relation $F_n = F_{n-1} + F_{n-2}$
, the second Fibonacci number is then 1 and adding an element now yields a tree 
of height 2, as the top of the tree is full. Since the second Fibonacci number 
still lower bounds the height of the tree, the base case does hold.\\[5pt]

\noindent \textbf{Inductive step}\\[5pt]
At every level of height i of an AVL tree with $\lambda_i$ elements, the height 
is one higher than its deepest subtree, and maximizing the allowed difference,
the other subtree must be of height two less than height i. By the inductive 
hypothesis, the number of elements of subtrees is given by Fibonacci number $i-1$
and $i-2$, thus the number of elements is number of elements of the subtrees plus
itself.
\begin{align*}
\lambda_i &= F_{i-1} + F_{i-2} + 1 \\
          &< F_{i-1} + F_{i-2} = F_i \Rightarrow \\
          \lambda_i &= \Omega (F_i) \Rightarrow \\
\end{align*}
\hfill $\square$

We know that the n'th Fibonacci number can be calculated using the Golden Ratio
in the relation \\

\begin{align*}
F_n =  \left\lfloor \frac{\phi^n}{\sqrt{5}} + \frac{1}{2} \right\rfloor
\end{align*}

We can omit the floor function since it only counts as a constant decimal 
difference, aswell as all the other constants and lower order terms. We get:

\begin{align*}
\lambda_i &= \Omega \left( \phi^n \right)
\end{align*}

\noindent
Taking the natural logarithm and shifting the asymptotic growth, we get:

\begin{align*}
n = O(\ln \lambda_i)
\end{align*}

\hfill $\square$

\newpage
\noindent \large{\textbf{b} \mdseries To insert into an AVL tree, we first
place a node into the appropriate place in binary search tree order.
Afterward, the tree might no longer be height balanced. Specifically, the
heights of the left and right children of some node might differ by 2.
Describe a procedure \texttt{Balance}, which takes a subtree rooted at $x$
whose left and right children are height balanced and have heights that differ
by at most 2, i.e., $|x.right.h - x.left.h| \leq 2$, and that alters the
subtree rooted at $x$ to be height balanced. (Hint: Use rotations)}
\\
\begin{algorithm}
	\SetKwInOut{Input}{Input}
	\SetKwInOut{Output}{Output}
	\SetKwFunction{Balance}{Balance}
	\BlankLine
	
	\Balance($T$) \\
	\Begin
	{
		\If{$x.left.h - x.right.h > 1$}
		{
			
		}
	}
\end{algorithm}
\\\\
\noindent \large{\textbf{c} \mdseries Using part (b), describe a recursive
procedure \texttt{AVL-Insert($x$,$z$)} that takes a node $x$ within an AVL
tree and a newly created node $z$ (whose key has already been filled in), and
adds $z$ to the subtree rooted at $x$, maintaining the property that $x$ is
the root of an AVL tree. As in \texttt{Tree-Insert} from section 12.3, assume
that $z.key$ has already been filled in and that $z.left =$ \texttt{NIL} and
$z.right =$ \texttt{NIL}; also assume that $z.h = 0$. Thus, to insert the node
$z$ into an AVL tree $T$, we call \texttt{AVL-Insert($T.root$,$z$)}.}
\\
...
\\\\
\noindent \large{\textbf{d} \mdseries Show that \texttt{AVL-Insert}, run on an
$n$-node AVL tree, takes $O(lg n)$ time and performs $O(1)$ rotations.}
\\
...


%========== optional ==========%

% no optional hand-ins


%========== extras ==========%

% no extra hand.ins

%========== exercises ==========%

% \newpage
% \pagestyle{fancy}
% \section*{Exercises}
% ...

\end{document}

