%	
%	solution.tex - Week 4, deadline on 24th of May 2013
%	
%	This document provides answers to the exercises and problems as defined by
%	the course for the week shown below.
%	
%	Mandatory:	16-1
%	Optional:	...
%	Extras:		...
%	

\documentclass[11pt,english]{article}

\usepackage[utf8]{inputenc}
\usepackage{fancyhdr}
\usepackage{sectsty}
\usepackage{amsmath,amssymb}	% for mathematical notation
\usepackage[linesnumbered]{algorithm2e}

%========== meta data ==========%

\title
{
	\vspace{1in}
	Algorithms \& Datastructures\\
	\huge Assignment 4
}

\author
{
	Casper B. Hansen\\
	\small Department of Computer Science\\
	\small The University of Copenhagen\\
	\texttt{fvx507@alumni.ku.dk}
	\and
	Hans J. T. Stephensen\\
	\small Department of Computer Science\\
	\small The University of Copenhagen\\
	\texttt{xkv467@alumni.ku.dk}
}

\date{\today}


%========== settings ==========%

\setlength{\headheight}{15pt}
\sectionfont{\Large}


%========== macros ==========%

% no macros yet


%========== document ==========%

\begin{document}

\clearpage
\maketitle
\thispagestyle{empty}

%========== mandatory ==========%

\newpage
\pagestyle{fancy}

\section*{Mandatory Hand-ins}

\subsection*{16-1 Coin Changing}
\large{Consider the problem of making change for $n$ cents using the fewest
number of coins. Assume that each coin's value is an integer.}
\\\\
\noindent \large{\textbf{a} \mdseries Describe a greedy algorithm to make
change consisting of quarters, dimes, nickels, and pennies. Prove that your
algorithm yields an optimal solution.}

\subsubsection*{Algorithm}
\noindent We will assume that the sequence $S$ given in the following
algorithm is sorted in decreasing order before being passed as an argument.

\begin{algorithm}
	\SetKwInOut{Input}{Input}
	\SetKwInOut{Output}{Output}
	\SetKw{KwDownTo}{down to}
	\SetKwFunction{Coin}{Coin}
	
	\Input{An array $S$ of coin denominators, sorted in decreasing order, and
an integer $n$ of cents.}
	\Output{An ordered sequence of pairs, where the first element in each pair
denotes the denominator in $S$, and the second element denotes the amount
required of that coin denominator.}
	\BlankLine
	
	$k = 1$\\
	$c = 0$\\
	Let $A$ be a new list
	
	\While{$k < S$.length}
	{
		\eIf{$n \geq S[k]$}
		{
			$c = c + 1$\\
			$n = n - S[k]$
		}
		{
			\If{$c \neq 0$}
			{
				$A$.append( ($S[k]$, $c$) )\\
			}
			$c = 0$\\
			$k = k + 1$
		}
	}
	\Return $A$
\end{algorithm}

\subsubsection*{Proving the optimal solution}
\noindent The above pseudo-code describes procedure of, on a given amount 
of money denoted by $n$, make the greedy choice by choosing the largest 
possible coin that's smaller than $n$. \\
The algorithm does not work for any choice of denominations, as will be 
shown in sub-problem c. However, it does work for this specific set of 
coins. The reason it works for this set of coins, is not very easy to 
see. \\
Any given amount of money is given by linear combinations of the coins. 
So an amount of money $y$ is given by.
\begin{align*}
y &= x_0 + 5x_1 + 10x_2 +25x_3
\end{align*}
Suppose that the dime isn't in the set of denominations $S$, that is $S' =
S{\ }\backslash \{10\}$. We see that the set $S'$ is purely based on powers
of 5. For such a set $S'$, we formally prove the optimal solution in (b),
which we will then refer to - we set $c=5$ and $k=2$ for this particular case.
Note that a dime is simply another way of expressing two nickels, that is to
say that adding the dime back into the set of denominations doesn't hinder
nor lessens an optimal solution, rather in strengthens it.
\\\\
\noindent \large{\textbf{b} Suppose that the available coins are in the
denominations that are powers of $c$, i.e., the denominations are $c^0$,
$c^1$,\dots, $c^k$ for some integers $c > 1$ and $k \geq 1$. Show that
the greedy algorithm always yields an optimal solution.}
\\\\
For any set of coin denominations given by $c^0$, $c^1$,\dots, $c^k$, where
$c > 1$ and $k \geq 1$, we observe that this forms the basis of a number
system. That is, the sum of all $c^i$, for $0 \leq i \leq k$, times $c - 1$
plus $1$ is equal to $c^{i+1}$, from which it then naturally follows that we
can produce any number bounded by $c^k$ using a combination of these denominators.

More formally, we state that
\begin{align}
	(c - 1)\sum_{i=0}^{k} c^i + 1 = c^{k+1}
\end{align}
And since we know that we can produce any number bounded by $c^k$, we must
now show that we can produce any number above this bound. We do this
inductively by giving the base case
\begin{align}
	c^k + 1 &= \sum_{i=0}^{k-1} a_i c^i + 1 \nonumber
	\quad \text{, where }a_i = c - 1\text{ is the amount of coin $c^i$.} \\
	& = (a_0 + 1)c^0 + a_1 c^1 + \dots + a_{k-1} c^{k-1}
\end{align}
which is above the previously mentioned bound, yet is still given by a
linear combination of the denominators. Since $c^0 = 1$ adding 1 is as simple
as incrementing the value of $a_0$. This observation leads into that when
$a_i < c - 1$, we simply add a coin of $c^i$. However, if $a_i = c - 1$, we
have that $(c - 1 + 1)c^i = c^{i+1}$. In other words assume that $a_i = c$,
then $c \cdot c^i = c^{i+1}$ which means that we can express the amount of
$c^i$ by means of $c^{i+1}$ - this then cascades.

As we examine the inductive step, we observe that we are simply counting on
top, and adding this count to as coefficients on any $c^i$, where
$0 \leq i \leq k$, as shown below.
\begin{align}
	c^{k+1} &= (c - 1)\sum_{i=0}^{k}{c^i} + 1\\
	&= \sum_{i=0}^{k+1}{c^i} \nonumber
\end{align}
It then follows by the induction, that we can produce any number since, for
any $c$ we might choose and $k \geq 0$, and $ac^0$ can produce any number.

Now that we have proved that we can produce any number using these denomitors
for any value of $c$ and $k$, we will now extend this to prove that the greedy
choice (locally optimal solution) always leaves a subproblem that exhibits the
optimal substructure property.

Consider the coin of highest value $C_h$ in the set $S$. Our algorithm makes
the assumption that as long as we can choose $C_h$ (that is, we make a greedy
choice), doing so yields a optimal substructure. Let's consider that this is
not the case, and choosing a coin of lower value $C_l$. By the definition of
the set $c^i | c^j$, where $i < j$, since the set is defined as powers of $c$.
It then follows that any $C_l$ divides $C_h$, meaning that we can represent
$C_h$ as $aC_l$, which contradicts the assumption that $C_l$ yields globally
optimal solution, since we must use $a$ of coin $C_l$, whereas we only need
one of $C_h$. This proves the greedy choice is always the best choice.

Let us now assume that we have exhausted the amount of $C_h$ we can choose,
we then remove it from the set $S$. Doing so leaves a set where $C_h$ is
abscent, which makes another coin $C_h'$ the most valuable, which in turn is
the next best choice, proving the optimal substructure property. Moreover,
since we assume that a penny, or $c^0$, is present in the set $S$ and that
$n$ is an integer (cents), that is we cannot have fractions of cents, the
algorithm must bottom out eventually - input of $n = 0$ is simply a linear
combination of the denominations where all leading constants are zero.
\\\\
\noindent \large{\textbf{c} Give a set of coin denominations for which the greedy algorithm
does not yield an optimal solution. Your set should include a penny, so
that there is a solution for every value of $n$.}
\\\\
For any set of coin denominations
$\{d \in \mathbb{Z}^{+} : d_i | d_{j}\}$, where $1 < i < j$, we must be able to
produce an optimal solution, because we cannot make any greedy choice that
violates an optimal solution of the following subproblem. If such a set of
coin denominations should not follow this definition, then we cannot maintain
that the algorithm always yields an optimal solution. Ie. given the set
$d = \{36, 20, 1\}$ of coin denominations, and the number of cents $n = 40$,
then the algorithm would produce the non-optimal solution $\{36, 1, 1, 1, 1\}$
where an optimal solution would actually be $\{20, 20\}$ - hence the greedy
choice fails to produce an optimal solution.
\\\\
\noindent \large{\textbf{d} Give an $O(nk)$-time algorithm that makes change for any set of $k$
different coin denominations, assuming that one of the coins is a penny.}
\\\\
% As the algorithm given in (a) runs in $O(nk)$-time, as we will prove now, we
% simply refer to this algorithm.
%
% On lines 1 through 3 we simply initialize some data, all of which takes
% constant time $\Theta(1)$. $k$ denotes the current index of $S$ we are
% considering to add to the result set, and we set this to be the first index
% initially. $c$ denotes the amount we are collecting of a particular index of
% $S$, that is how many need want of $S[k]$ in our result set. The \texttt{
% while}-loop on lines 4 through 15 contributes $O(nk)$-time (note that when we
% $k$ here we aren't referring to the algorithms internal variable, but rather
% the length of $S$), as it run at most $k$ iterations. To see why it runs in
% $O(nk)$-time, note that $k$ (the conditional termination variable) is only
% incremented when certain conditions are true. Lines 5-7 collects one of $S[k]$
% if $n \geq S[k]$, and simply reiterates this until $n < S[k]$, hence the
% $O(nk)$ contribution of the \texttt{while}-loop. Whenever $n$ becomes less
% than $S[k]$ an iteration over the $k$'s is done, and so we add the collected
% coins to the set on lines 9-12, and increment $k$ on line 13. Line 16 simply
% returns the result set.


%========== optional ==========%

% no optional hand-ins


%========== extras ==========%

% no extra hand.ins

%========== exercises ==========%

% \newpage
% \pagestyle{fancy}
% \section*{Exercises}
% ...

\end{document}

