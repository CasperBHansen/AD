%	
%	solution.tex - Week 4, deadline on 24th of May 2013
%	
%	This document provides answers to the exercises and problems as defined by
%	the course for the week shown below.
%	
%	Mandatory:	16-1
%	Optional:	...
%	Extras:		...
%	

\documentclass[11pt,english]{article}

\usepackage[utf8]{inputenc}
\usepackage{fancyhdr}
\usepackage{sectsty}
\usepackage{amsmath,amssymb}	% for mathematical notation
\usepackage[linesnumbered]{algorithm2e}

%========== meta data ==========%

\title
{
	\vspace{1in}
	Algorithms \& Datastructures\\
	\huge Assignment 4
}

\author
{
	Casper B. Hansen\\
	\small Department of Computer Science\\
	\small The University of Copenhagen\\
	\texttt{fvx507@alumni.ku.dk}
	\and
	Hans J. T. Stephensen\\
	\small Department of Computer Science\\
	\small The University of Copenhagen\\
	\texttt{xkv467@alumni.ku.dk}
}

\date{\today}


%========== settings ==========%

\setlength{\headheight}{15pt}
\sectionfont{\Large}


%========== macros ==========%

% no macros yet


%========== document ==========%

\begin{document}

\clearpage
\maketitle
\thispagestyle{empty}

%========== mandatory ==========%

\newpage
\pagestyle{fancy}

\section*{Mandatory Hand-ins}

\subsection*{16-1 Coin Changing}
\large{Consider the problem of making change for $n$ cents using the fewest
number of coins. Assume that each coin's value is an integer.}
\\\\
\noindent \large{\textbf{a} \mdseries Describe a greedy algorithm to make
change consisting of quarters, dimes, nickels, and pennies. Prove that your
algorithm yields an optimal solution.}

\subsubsection*{Algorithm}
\noindent We will assume that the sequence $S$ given in the following
algorithm is sorted in decreasing order before being passed as an
argument. Should we need to make up for this assumption, ... yada yada

\begin{algorithm}
	\SetKwInOut{Input}{Input}
	\SetKwInOut{Output}{Output}
	\SetKw{KwDownTo}{down to}
	
	\Input{A sequence $S$ of coin denominators, sorted in decreasing order,
and a number $n$ of cents.}
	\Output{An ordered sequence of pairs, where the first element in each pair
denotes the denominator by its index in $S$, and the second element denotes
the amount required of that coin denominator.}
	\BlankLine
	
	$k = 1$\\
	$c = 0$\\
	Let $A$ be a new list
	
	\While{$k < S$.length}
	{
		\eIf{$n \geq S[k]$}
		{
			$c = c + 1$\\
			$n = n - S[k]$
		}
		{
			\If{$c \neq 0$}
			{
				$A$.append( ($k$, $c$) )\\
			}
			$c = 0$\\
			$k = k + 1$
		}
	}
	
	\Return $A$	
\end{algorithm}

\subsubsection*{Proving the optimal solution}
\noindent The above pseudo-code describes procedure of, on a given amount 
of money denoted by $n$, make the greedy choice by choosing the largest 
possible coin that's smaller than $n$. \\
The algorithm does not work for any choice of denominations, as will be 
shown in sub-problem c. However, it does work for this specific set of 
coins. The reason it works for this set of coins, is not very easy to 
see. \\
Any given amount of money is given by linear combinations of the coins. 
So an amount of money $y$ is given by.
\begin{align*}
y &= x_0 + 5x_1 + 10x_2 +25x_3
\end{align*}
Any value of $y < 5$, it is trivially seen, that only the first coin (the
penny) can be used. with $5 \leq y < 10$, it can also easily be seen that
one coin of five (the nickle) must be used, reducing the problem to size
$y < 5$ again. For values $10 \leq y < 25$ it is again pretty easy to see
that 10 (the dime) being a multiple of 5, must be part of the result. \\
For values above 25 which is not a multiple of either 5 or 10 alone, but 
linear combinations of both, the argument is not so easy. since a value 
of size 50, is two times the problem of 25, we only need to check the 
values of $\{ 26, 27, \cdots, 48, 49 \}$.
\\\\
\noindent \large{\textbf{b} Suppose that the available coins are in the
denominations that are powers of $c$, i.e., the denominations are $c^0$,
$c^1$,\dots, $c^k$ for some integers $c > 1$ and $k \geq 1$. Show that
the greedy algorithm always yields an optimal solution.}
% think c-nary representation, DiMS book as reference!
\\\\
For any set of coin denominations given by $c^0$, $c^1$,\dots, $c^k$, where
$c > 1$ and $k \geq 1$, we observe that this forms the basis of a number
system. That is, the sum of all $c^i$, for $0 \leq i \leq k$, plus $1$ is
equal to $c^{i+1}$, from which it then follows that we can produce any number
using a combination of these denominators.

More formally, we state that
\begin{align}
	\sum_{i=0}^{j} c{_i}^{j} + 1 = c^{j+1}
\end{align}
And since we know that we can produce any number bounded by $c^k$, we must
now show that we can produce any number above this bound. We do this
inductively giving the base case
\begin{align}
	c^k + 1 &= c^0 + c^1 + \dots + c^k + 1\\
	&= 2c^0 + c^1 + \dots + c^k
\end{align}
which is above the bound previously mentioned bound, yet is still given by a
combination of the denominators. As we examine the inductive step, we observe
that we are simply counting on top, and adding this count to as coefficients
on any $c^i$, where $0 \leq i \leq k$, as shown below.
\begin{align}
	c^{k+1} &= (c - 1)\sum_{i=0}^{k}{c^i} + 1\\
	&= \sum_{i=0}^{k+1}{c^i} \nonumber
\end{align}
It then follows by the induction, that we can produce any number.

\subsubsection*{A set that fails}
\large{Give a set of coin denominations for which the greedy algorithm
does not yield an optimal solution. Your set should include a penny, so
that there is a solution for every value of $n$.}
\\\\
For any set of coin denominations
$\{d \in \mathbb{Z}^{+} : d_i | d_{i+1}\}$, we must be able to produce an
optimal solution, because we cannot make any greedy choice that violates an
optimal solution of the following subproblem. If such a set of coin
denominations should not follow this definition, then we cannot maintain
that the algorithm always yields an optimal solution. Ie. given the set
$d = \{36, 20, 1\}$ of coin denominations, and the number of cents $n = 40$,
then the algorithm would produce the non-optimal solution $\{36, 1, 1, 1, 1\}$
where an optimal solution would actually be $\{20, 20\}$ - hence the greedy
choice fails to produce an optimal solution.

\subsubsection*{An $O(nk)$-time algorithm}
\large{Give an $O(nk)$-time algorithm that makes change for any set of $k$
different coin denominations, assuming that one of the coins is a penny.}
% move (a) down here, and just describe in (a) instead?
\\\\
...
\begin{algorithm}
	\SetKwInOut{Input}{Input}
	\SetKwInOut{Output}{Output}
	\SetKw{KwDownTo}{down to}
	\SetKw{Nil}{NIL}
	\SetKwFunction{Coin}{Coin}

	\Input{A sequence $S$ of coin denominators, sorted in decreasing order,
and a number $n$ of cents. To avoid a global variable we pass a cache $table$
in each recursive call.}
	\Output{An ordered sequence of pairs, where the first element in each pair
denotes the denominator by its index in $S$, and the second element denotes
the amount required of that coin denominator.}
	\BlankLine
	
	\If{$table = $ \Nil}
	{
		$table$ is created as a new hash table.
	}

	\uIf{$n < S[1]$}
	{
		\Return An array of length $n$ filled with $1$'s
	}
	\uElseIf{$n$ is in $S$}
	{
		\Return [$n$]
	}
	\Else
	{
		s = $n + 1$\\
		c = []
		
		\For{$i = 0$ \KwTo $n - 1$}
		{
			\eIf{$i + 1$ is in the $table.keys$}
			{
				$a = table[i+1]$
			}
			{
				$a = $ \Coin{$S$, $i + 1$, $table$}\\
				$table[i+1] = a$
			}
			\eIf{$n - 1 - i$ is in the $table.keys$}
			{
				$b = table[n - 1 - i]$
			}
			{
				$b = $ \Coin{$S$, $n - 1 - i$, $table$}\\
				$table[n - 1 - i] = b$
			}
			$l = a.length + b.length$
			\If{$l < s$}
			{
				$s = l$\\
				$c = a + b$
			}
		}
	}
	\Return $c$
\end{algorithm}
...

%========== optional ==========%

% no optional hand-ins


%========== extras ==========%

% no extra hand.ins

%========== exercises ==========%

% \newpage
% \pagestyle{fancy}
% \section*{Exercises}
% ...

\end{document}

