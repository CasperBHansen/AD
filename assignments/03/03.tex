%	
%	solution.tex - Week 3, deadline on 17th of May 2013
%	
%	This document provides answers to the exercises and problems as defined by
%	the course for the week shown below.
%	
%	Mandatory:	15-2
%	Optional:	...
%	Extras:		...
%	

\documentclass[11pt,english]{article}

\usepackage[utf8]{inputenc}
\usepackage{fancyhdr}
\usepackage{sectsty}
\usepackage{amsmath,amssymb}	% for mathematical notation
\usepackage[linesnumbered]{algorithm2e}

%========== meta data ==========%

\title
{
	\vspace{1in}
	Algorithms \& Datastructures\\
	\huge Assignment 3
}

\author
{
	Casper B. Hansen\\
	\small Department of Computer Science\\
	\small The University of Copenhagen\\
	\texttt{fvx507@alumni.ku.dk}
	\and
	Hans J. T. Stephensen\\
	\small Department of Computer Science\\
	\small The University of Copenhagen\\
	\texttt{xkv467@alumni.ku.dk}
}

\date{\today}


%========== settings ==========%

\setlength{\headheight}{15pt}
\sectionfont{\Large}


%========== macros ==========%

% no macros yet


%========== document ==========%

\begin{document}

\clearpage
\maketitle
\thispagestyle{empty}

%========== mandatory ==========%

\newpage
\pagestyle{fancy}

\section*{Mandatory Hand-ins}

\subsection*{15-2 Longest Palindrome Subsequence}
\large{A palindrome is a nonempty string over some alphabet that reads the
forward and backward. Examples of palindromes are all strings of length 1,
\textit{civic}, \textit{racecar}, and \textit{aibohphobia} (fear of
palindromes).

Give an efficient algorithm to find the longest palindrome that is a
subsequence of a given input string. For example, given the input
\texttt{character}, your algorithm should return \texttt{carac}. What is the
running time of your algorithm?}

\subsubsection*{Algorithm}
We will regard the input sequence $S$ as an object that provides us with a
\texttt{length} method. That is, calling this method on the sequence object
returns the length of the sequence, just as it would if it was an array.

We shall assume that we have access to an \texttt{LCS} procedure, as given in
the CLRS book, pages 394-395, with the minor difference being that we assume
it to return the actual sequence being computed by it, rather than what it
actually does in the books version of the algorithm.

Furthermore, we also assume that we have access to the simple math functions
\texttt{Floor} and \texttt{Ceil}, and a helper function \texttt{Reverse} that
we will regard as taking $\Theta(n)$ as it simply copies elements from one
sequence into another.

\begin{algorithm}
	\SetKwInOut{Input}{Input}
	\SetKwInOut{Output}{Output}
	\SetKwFunction{Reverse}{Reverse}
	\SetKwFunction{LCS}{LCS}
	\SetKwFunction{Floor}{Floor}
	\SetKwFunction{Ceil}{Ceil}
	\SetKw{KwDownTo}{down to}
	
	\Input{Takes a sequence $S$ of characters.}
	\Output{The longest palindrome subsequence of the input sequence.}
	\BlankLine
	
	$R =$ \Reverse($S$)\\
	$s =$ \LCS($S$, $R$)\\
	$l =$ $s$.length / 2\\
	Let $out$ be a new string
	
	Copy $s[1 \dots l]$ into $out$\\
	\If{l > \Floor(l)}
	{
		i = \Ceil(l)\\
		Append $s_i$ to $out$
	}
	Append $s[l+1 \dots s\text{.length}]$
	
	\Return $out$	
\end{algorithm}

\subsubsection*{Analysis}
During initialization of the internal variables within the algorithm we spend
$\Theta(n)$ on reversing the original string $S$ and storing it in $R$ on line
1, the following line 2 utilizes the \texttt{LCS} function, which contributes
$\Theta(mn)$, but since our inputs are equal in size, that is $m = n$, it then
becomes $\Theta(n^2)$. On line 3 we spend constant time $\Theta(1)$
calculating the floor of the length of the longest common subsequence $s$ over
2. We then spend $\Theta(n)$ on copying the first half of $s$ into the $out$
sequence, and then if the sequence $s$ is of unequal length we add the
following symbol of the sequence $s$ to $out$, and lastly we copy the first
half of $s$ as already copied, but in reverse order to $out$.

We then have that $T(n) = \Theta(n^2) + \Theta(2n) + \Theta(1) = \Theta(n^2)$.

\subsubsection*{Proof}
%Since the \texttt{LCS} procedure calculates the longest subsequence, and that
%the longest subsequence of a sequence $S$ and its reverse $R$ must contain all
%the symbols of all the contributing parts of a \textit{LPS}\footnote{longest
%palindromic subsequence}, we can then argue that the first half, including any
%middle symbol in odd-numbered sequence must be valid for the first half.
%
%The problem that arises, and that we must argue we have resolved in our
%algorithm, is that the \texttt{LCS} of $S$ and $R$, as defined above, won't
%necessarily be an \textit{LPS}, but since the path in the matrix should be
%symmetric we can argue that if we trace back through the first half we have
%already added, then the reverse of those must inevitably be a path to be found
%in the matrix moving forward from this middle symbol. From this, we claim that
%it is simply a matter of copying the reverse of the first half onto the out
%(excluding the middle symbol), and this gives us the \textit{LPS}, the desired
%result.

Assume that there exists no path in the matrix, as calculated by \texttt{LCS},
such that reversed sequence $R$ of the first half of a sequence $S$ is a legal
path option, as defined by the \texttt{LCS} procedure. Since the input
sequences are each others inverse, then the matrix must be symmetrical, and
this contradicts our assumption. Therefore, any valid half of the \texttt{LCS}
must have a symmetric inverse path, and for odd-numbered length sequences,
since any sequence of length 1 is a palindrome in and of itself, then
stitching it in-between the previously mentioned must still yield a valid
palindromic subsequence.


%========== optional ==========%

% no optional hand-ins


%========== extras ==========%

% no extra hand.ins

%========== exercises ==========%

% \newpage
% \pagestyle{fancy}
% \section*{Exercises}
% ...

\end{document}

