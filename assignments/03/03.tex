%	
%	solution.tex - Week 3, deadline on 17th of May 2013
%	
%	This document provides answers to the exercises and problems as defined by
%	the course for the week shown below.
%	
%	Mandatory:	15-2
%	Optional:	...
%	Extras:		...
%	

\documentclass[11pt,english]{article}

\usepackage{fancyhdr}
\usepackage{sectsty}
\usepackage{amsmath,amssymb}	% for mathematical notation
\usepackage[linesnumbered]{algorithm2e}

%========== meta data ==========%

\title
{
	\vspace{1in}
	Algorithms \& Datastructures\\
	\huge Assignment 3
}

\author
{
	Casper B. Hansen\\
	\small Department of Computer Science\\
	\small The University of Copenhagen\\
	\texttt{fvx507@alumni.ku.dk}
	\and
	Hans J. T. Stephensen\\
	\small Department of Computer Science\\
	\small The University of Copenhagen\\
	\texttt{xkv467@alumni.ku.dk}
}

\date{\today}


%========== settings ==========%

\setlength{\headheight}{15pt}
\sectionfont{\Large}


%========== macros ==========%

% no macros yet


%========== document ==========%

\begin{document}

\clearpage
\maketitle
\thispagestyle{empty}

%========== mandatory ==========%

\newpage
\pagestyle{fancy}

\section*{Mandatory Hand-ins}

\subsection*{15-2 Longest Palindrome Subsequence}
\large{A palindrome is a nonempty string over some alphabet that reads the
forward and backward. Examples of palindromes are all strings of length 1,
\textit{civic}, \textit{racecar}, and \textit{aibohphobia} (fear of
palindromes).

Give an efficient algorithm to find the longest palindrome that is a
subsequence of a given input string. For example, given the input
\texttt{character}, your algorithm should return \texttt{carac}. What is the
running time of your algorithm?}

\subsubsection*{Algorithm}
We will regard the input string $S$ as an object that provides us with a
\texttt{length} method. That is, calling this method on the string object
returns the length of the string, just as it would if it was an array.
\begin{algorithm}
	\SetKwInOut{Input}{Input}
	\SetKwInOut{Output}{Output}
	
	\Input{Takes a string $S$}
	\Output{...}
	\BlankLine
	\Return 0
	
\end{algorithm}

\subsubsection*{Analysis}
...

%========== optional ==========%

% no optional hand-ins


%========== extras ==========%

% no extra hand.ins

%========== exercises ==========%

% \newpage
% \pagestyle{fancy}
% \section*{Exercises}
% ...

\end{document}

