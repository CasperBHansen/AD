%	
%	disposition.tex
%	
%	This document provides a summary of all of the exam question topics for
%	the algorithms and datastructures course at the university of copenhagen
%	department of computer science. It was written to serve as an excellent
%	asset for the oral exam preparation, with most of the subtopics covered
%	for any given exam topic in a concise and easily understandable format.
%	

\documentclass[11pt,english]{book}

\usepackage{fancyhdr}						% page styling
\usepackage{sectsty}						% section styling
\usepackage{amsmath,amssymb}				% mathematical notation
\usepackage[linesnumbered]{algorithm2e}		% algorithm pseudo-code

%========== meta data ==========%

\title
{
	\vspace{1in}
	Algorithms \& Datastructures\\
	\huge Disposition
}

\author
{
	Casper B. Hansen\\
	\small Department of Computer Science\\
	\small The University of Copenhagen\\
	\texttt{fvx507@alumni.ku.dk}
	\and
	Hans J. T. Stephensen\\
	\small Department of Computer Science\\
	\small The University of Copenhagen\\
	\texttt{xkv467@alumni.ku.dk}
}

\date{\today}


%========== settings ==========%

\setlength{\headheight}{15pt}
\sectionfont{\Large}


%========== macros ==========%

\newcommand{\sethead}[3]{\lhead{#1}\chead{#2}\rhead{#3}}
\newcommand{\setfeet}[3]{\lfoot{#1}\cfoot{#2}\rfoot{#3}}

%========== document ==========%

\begin{document}

\maketitle
\thispagestyle{empty}
%\tableofcontents

%========== preamble ==========%

\newpage
\pagestyle{fancy}

\chapter*{Preamble}
In this document we will discuss the most important principles of algorithms
and datastructures in accordance with the syllabus of the course under the
same title at the University of Copenhagen, Department of Computer Science.

\section*{About the authors}
We're a small group of Computer Science students, attending the course
mentioned above at the time of writing this document as collaborative effort
to make a great disposition for the course exam, that will both increase our
own understanding of the subjects discussed as well as produce an invaluable
resource for others attending the course, or who wish to learn about
algorithms and datastructures in generel.

\section*{Why we made this document}
The document is was written in order to support our own understanding of the
course contents as well as aid us during the exam preparation time.

\section*{How to use this document}
The intention of the document is \textit{not} to be a fullfledged introduction
to algorithms and datastructure that you can just pick up and start learning
from. As such it requires at least an accompanying book, or that you are
attending lectures on the subject, or some other means that provide a base
knowledge.

The document is merely a tool for looking up things quickly, and defines the
subjects in a concise manner, hence no in-depth discussion of any subject is
to be found in this document. The sole purpose of the document is precisely
that; a resource for quickly looking up the concepts discussed presented at
lectures, or in a book.

\section*{Recommendations}
As far as we know the book \textit{Introduction to Algorithms} by Thomas H.
Cormen, Charles E. Leiserson, Ronald L. Rivest and Clifford Stein, is pretty
much the go-to choice if you want to study algorithms, and since this was our
textbook during the course this is what we'll recommend for you.


%========== Chapter 1 - Definitions ==========%

\thispagestyle{fancyplain}

\chapter{Definitions}
\label{ch:definitions}
In this chapter we will discuss the most important definitions of algorithms
and datastructures.

\section{Asymptotic Notation}
\label{ch:definitions|sec:asymptotic-notation}
The analysis of the growth of a function is called asymptotic analysis. The
mathematical notation used to describe this analysis is called asymptotic
notation.

\subsection{$O$}
\label{ch:definitions|sec:asymptotic-notation|sub:big-o}
Defines a tight upperbounded growth of a function. That is, $O(n)$ is
analogous to the comparison $\leq$.

\subsection{$o$}
\label{ch:definitions|sec:asymptotic-notation|sub:little-o}
...

\subsection{$\Omega$}
\label{ch:definitions|sec:asymptotic-notation|sub:big-omega}
...

\subsection{$\omega$}
\label{ch:definitions|sec:asymptotic-notation|sub:litte-omega}
...

\subsection{$\Theta$}
\label{ch:definitions|sec:asymptotic-notation|sub:theta}
...

\section{Loop Invariant}
\label{ch:definitions|sec:loop-invariant}
A loop invariant is based on the notion of mathematical induction, as it
provides a proof of the algorithms correctness in all three of its executed
stages; initialization, maintenance and termination.

\subsection{Initialization}
...

\subsection{Maintenance}
...

\subsection{Termination}
...

\section{Summary}
\label{ch:definitions|sec:asymptotic-notation|sec:summary}
% TODO: list the definitions in a table

%========== Exam Topic 1 - Divide & Conquer ==========%

\chapter{Divide \& Conquer}
\label{ch:divideandconquer}
...


%========== Exam Topic 2 - Priority Queues ==========%

\chapter{Priority Queues}
\label{ch:priorityqueyes}
...


%========== Exam Topic 3 - Balanced Search B-Trees ==========%

\chapter{Balanced Binary Search Trees}
\label{ch:bbstrees}

\textbf{Relevant Assigment} Week ?, Problem ?-?\\\\
\textbf{Keywords} 
\vspace{1in}

\noindent Lorem ipsum dollarLorem ipsum dollar ....

%========== Chapter 4 - Dynamic Programming ==========%

\chapter{Dynamic Programming}
\label{ch:dynamicprog}

\textbf{Relevant Assignment} Week 3, Problem 15-2\\\\
\textbf{Keywords} Rod-cutting, fibonacci, time-memory trade-off
\vspace{1in}

\noindent There are two key characteristics that a problem must have for
dynamic programming to be a viable solution; optimal substructure and
overlapping subproblems.
\\\\
\noindent \textbf{Optimal Substructure}\\
... % p. 379
\\\\
\noindent \textbf{Overlapping Subproblems}\\
... % p. 384

\newpage
\section{Methods of Approach}
There are usually two equivalent ways to implement a dynamic-programming
approach; top-down with memoization and bottom-up.

\subsection{Top-down with Memoization}
...

\subsection{Bottom-up}
...


%========== Chapter 5 - Greedy Algorithms ==========%

\chapter{Greedy Algorithms}
\label{ch:greedyalg}
...


%========== Chapter 6 - Amortized Complexity ==========%

\chapter{Amortized Complexity}
\label{ch:amortized}
...


%========== Chapter 7 - Minimum Spanning Tree ==========%

\chapter{Minimum Spanning Trees}
\label{ch:minspantree}
...


%========== Chapter 8 - Shortest Path Algorithms ==========%

\chapter{Shortest Path Algorithms}
\label{ch:shortestpath}
...


%========== bibliography ==========%

\newpage

\sethead{}{Bibliography}{}

\begin{thebibliography}{9}
\label{bibliography}

\bibitem{lamport94}
  Thomas H. Cormen, Charles E. Leiserson, Ronald L. Rivest, Clifford Stein\\
  \textit{Introduction to Algorithms}\\
  MIT Press, 3rd. Edition, 2009\\

\end{thebibliography}


\end{document}

